% ADD REFERENCE TO THE SWARTHMORE COLLEGE
\documentclass{article}
\usepackage{amsmath}
\usepackage{array}

\begin{document}

\title{Modified Nodal Analysis}
\maketitle

\section{Circuit Notation}

For a circuit with \( n \) nodes, \( m \) voltage sources, and \( k \) independent current sources:

\begin{table}[h!]
\centering
\begin{tabular}{|r|c|}
\hline
\textbf{Circuit Component} & \textbf{Symbol} \\
\hline
Ground & 0 \\
Node & 1 $\to$ n \\
Nodal Voltage & \( v_1 \to v_n \) \\
Independent Voltage Source & \( V_1 \to V_m \) \\
Current through Independent Voltage Source & \( i_1 \to i_m \) \\
Independent Current Source & \( I_1 \to I_k \) \\
\hline
\end{tabular}
\end{table}

\section{Modified Nodal Analysis (MNA)}

MNA will reduce circuits that only have passive components and independent voltage or current sources into the form:

\[
\mathbf{Ax} = \mathbf{z}
\]

For a circuit with \( n \) nodes and \( m \) voltage sources:

\subsection{The $\mathbf{A}$ Matrix}
\begin{itemize}
    \item \textbf{Size}: \( (n + m) \times (n + m) \)
    \item Contains \textbf{4 sub-matrices}: the conductance matrix (\(\mathbf{G}\)), the voltage source matrices (\(\mathbf{B}\) and \(\mathbf{C}\)), and the dependent source matrix (\(\mathbf{D}\)).
\end{itemize}

\[
\begin{bmatrix}
\mathbf{G} &  \mathbf{B} \\
\mathbf{C} & \mathbf{D} \\
\end{bmatrix}
\]

\begin{itemize}
    \item \textbf{The $\mathbf{G}$ Matrix}:
    \begin{itemize}
        \item \textbf{Size}: \( (n \times n) \)
        \item Each diagonal term is equal to the sum of the conductance of elements connected to the corresponding node. \textit{Example}: The first diagonal term is the sum of conductances connected to node 1.
        \item Each off-diagonal term is the negative conductance of the element connected to the pair of corresponding nodes. \textit{Example}: A resistor connected to nodes 2 and 3 will be placed in the \(\mathbf{G}\) matrix at positions (2, 3) and (3, 2).
    \end{itemize}
    \item \textbf{The $\mathbf{B}$ Matrix}:
    \begin{itemize}
        \item \textbf{Size}: \( (n \times m) \)
        \item Contains only the values -1, 0, and 1.
        \item If the \(n\)th node is connected to the \(m\)th voltage source's positive terminal, then the element at (n, m) is 1.
        \item If the \(n\)th node is connected to the \(m\)th voltage source's negative terminal, then the element at (n, m) is -1.
        \item Otherwise, the entry is 0.
        % TODO: Need to address the $\mathbf{C}$ and $\mathbf{D}$ matrices for dependent sources
    \end{itemize}
    \item \textbf{The $\mathbf{C}$ Matrix}:
    \begin{itemize}
        \item \textbf{Size}: \( (m \times n) \)
        \item Transpose of the \(\mathbf{B}\) matrix.
    \end{itemize}
    \item \textbf{The $\mathbf{D}$ Matrix}:
    \begin{itemize}
        \item \textbf{Size}: \( (m \times m) \)
        \item Contains all 0s.
    \end{itemize}
\end{itemize}

\subsection{The $\mathbf{x}$ Vector}
\begin{itemize}
    \item \textbf{Size}: \( (n + m) \times 1 \)
    \item Contains \textbf{2 vectors}: \(\mathbf{v}\) and \(\mathbf{j}\)
\end{itemize}

\[
\mathbf{x} = 
\begin{bmatrix}
\mathbf{v} \\
\mathbf{j} \\
\end{bmatrix}
\]

\begin{itemize}
    \item \textbf{The $\mathbf{v}$ Vector}:
    \begin{itemize}
        \item Each entry of the vector is the node voltage of the \(n\)th node (no entry for ground, node 0).
        \item \textit{Example}: 
        \[
        \mathbf{v} = 
        \begin{bmatrix}
        v_1 \\
        \vdots \\
        v_n \\
        \end{bmatrix}
        \]
    \end{itemize}
    \item \textbf{The $\mathbf{j}$ Vector}:
    \begin{itemize}
        \item Each entry of the vector is the current flowing into the \(m\)th voltage source.
        \item \textit{Example}: 
        \[
        \mathbf{j} = 
        \begin{bmatrix}
        i_1 \\
        \vdots \\
        i_m \\
        \end{bmatrix}
        \]
    \end{itemize}
\end{itemize}

\subsection{The $\mathbf{z}$ Vector}
\begin{itemize}
    \item \textbf{Size}: \( (n + m) \times 1 \)
    \item Contains \textbf{2 vectors}: \(\mathbf{i}\) and \(\mathbf{e}\)
\end{itemize}

\[
\mathbf{z} = 
\begin{bmatrix}
\mathbf{i} \\
\mathbf{e} \\
\end{bmatrix}
\]

\begin{itemize}
    \item \textbf{The $\mathbf{i}$ Vector}:
    \begin{itemize}
        \item \textbf{Size}: \( (n \times 1) \)
        \item The \(n\)th element is the sum of the current sources into the \(n\)th node (Node 0 isn't included). If no current source is connected to the \(n\)th node, then \((n, 1) = 0\).
    \end{itemize}
    \item \textbf{The $\mathbf{e}$ Vector}:
    \begin{itemize}
        \item \textbf{Size}: \( (m \times 1) \)
        \item The \(m\)th entry contains the value of the \(m\)th voltage source.
    \end{itemize}
\end{itemize}

\end{document}
